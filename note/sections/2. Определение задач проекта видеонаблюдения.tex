\section{Определение задач проекта видеонаблюдения}

Система видеонаблюдения предназначена для выполнения основных задач:

\begin{enumerate}
    \item Повышение эффективности обеспечения пропускного и внутри объектового режима, антитеррористической защищённости;
    \item Обеспечение и контроль качества и безопасности деятельности внутри объекта;
    \item Оценка и контроль данных обстановки, анализ обстановки, принятых мер по ликвидации нештатных ситуации, уточнение и корректировка по обстановке заранее разработанных вариантов решений по ликвидации каждой типовой нештатной ситуации;
    \item Получение информации о текущем состоянии объектов, анализ и оценка достоверности поступившей информации, доведение ее до руководства;
    \item Обеспечение противопожарной защиты помещений, зданий, сооружений и территории;
    \item Повышение эффективности действий при возникновении нештатных и чрезвычайных ситуаций, оповещения студентов (получателей/потребителей образовательных услуг) и работников об угрозе возникновения чрезвычайных ситуаций, необходимых действиях по эвакуации;
    \item Отслеживание, фиксация обстоятельств и фактов в целях недопущения и минимизации рисков материальных потерь, имущества в УрТИСИ СибГУТИ, сохранность личного имущества работников, имущества контрагентов и студентов (получателей/потребителей образовательных услуг), несанкционированного проникновения в служебные помещения, ущерба здоровью людей;
    \item Фиксация возможных противоправных действий, которые могут нанести вред имуществу. В случае необходимости материалы видеозаписей, полученных камерами видеонаблюдения, могут быть переданы и использованы в качестве доказательства соответствующими службами и государственными органами в случаях, предусмотренных действующим законодательством РФ для доказывания факта совершения противоправного действия, а также для установления личности лица, совершившего соответствующее противоправное действие;
    \item Осуществление контроля трудовой дисциплины и обеспечение объективности при вынесении дисциплинарных взысканий для доказывания факта совершения дисциплинарного проступка работником;
    \item Осуществление контроля в условиях, где другими средствами обеспечить его невозможно;
    \item Предоставление информации по официальным запросам соответствующих служб и государственных органов в случаях, предусмотренных действующим законодательством РФ;
\end{enumerate}
