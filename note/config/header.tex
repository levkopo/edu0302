%%% Здесь выбираются необходимые графы
\documentclass[hpadding=3mm,russian,utf8,pointsection,simple]{styles/eskdtext}
\usepackage{fontspec}
\defaultfontfeatures{Mapping=tex-text} % Для того чтобы работали стандартные сочетания символов ---, --, << >> и т.п.

%%% Что бы работал eskdx и некоторые другие пакеты LaTeX
\usepackage{xecyr}

%%% Для работы шрифтов
\usepackage{xunicode}
\usepackage{xltxtra}

%%% Для работы с русскими текстами (расстановки переносов, последовательность комманд строго обязательна)
% \usepackage{polyglossia}
% \setdefaultlanguage{russian}
%\newfontfamily{\cyrillicfontt}{GOST_B}
%\set{GOST_type_A}
%\setmainfont{GOST_type_A.ttf}
%\setromanfont{GOST_type_A.ttf}
%\setsansfont{GOST_type_A.ttf}
%\setmonofont{GOST_type_A.ttf}
%
\setmainfont[
  BoldFont={Times_New_Roman_Bold.ttf},
  ItalicFont={Times_New_Roman_Italic.ttf},
  BoldItalicFont={Times_New_Roman_Bold_Italic.ttf}
]{Times_New_Roman.ttf}
\setromanfont[
  BoldFont={Times_New_Roman_Bold.ttf},
  ItalicFont={Times_New_Roman_Italic.ttf},
  BoldItalicFont={Times_New_Roman_Bold_Italic.ttf}
]{Times_New_Roman.ttf}
\setsansfont[
  BoldFont={Times_New_Roman_Bold.ttf},
  ItalicFont={Times_New_Roman_Italic.ttf},
  BoldItalicFont={Times_New_Roman_Bold_Italic.ttf}
]{Times_New_Roman.ttf}
\setmonofont[
  BoldFont={Times_New_Roman_Bold.ttf},
  ItalicFont={Times_New_Roman_Italic.ttf},
  BoldItalicFont={Times_New_Roman_Bold_Italic.ttf}
]{Times_New_Roman.ttf}

% polyglossia only
% \newfontfamily\cyrillicfont{GOST_type_A} 
% \newfontfamily\cyrillicfontrm{GOST_type_A}
% \newfontfamily\cyrillicfonttt{GOST_type_A}
% \newfontfamily\cyrillicfontsf{GOST_type_A}
%\defaultfontfeatures{Mapping=tex-text}


%%% Для работы со сложными формулами
\usepackage{amsmath}
\usepackage{amssymb}

%%% Что бы использовать символ градуса (°) - \degree
\usepackage{gensymb}


%%% Для переноса составных слов
%\XeTeXinterchartokenstate=1
\XeTeXcharclass `\- 24
\XeTeXinterchartoks 24 0 ={\hskip\z@skip}
\XeTeXinterchartoks 0 24 ={\nobreak}

%%% Ставим подпись к рисункам. Вместо «рис. 1» будет «Рисунок 1»
\addto{\captionsrussian}{\renewcommand{\figurename}{Рисунок}}
%%% Убираем точки после цифр в заголовках
\def\russian@capsformat{%
  \def\postchapter{\@aftersepkern}%
  \def\postsection{\@aftersepkern}%
  \def\postsubsection{\@aftersepkern}%
  \def\postsubsubsection{\@aftersepkern}%
  \def\postparagraph{\@aftersepkern}%
  \def\postsubparagraph{\@aftersepkern}%
}



% Автоматически переносить на след. строку слова которые не убираются
% в строке
\sloppy

%%% Для вставки рисунков
\usepackage{graphicx}
\graphicspath{ {images/} }

%%% Для вставки интернет ссылок, полезно в библиографии
\usepackage{url}

%%% Подподразделы(\subsubsection) не выводим в содержании
\setcounter{tocdepth}{2}

%%% Большие таблицы
\usepackage{longtable}

%%% Отключение отступов в перечислениях
%\usepackage{enumitem}
%\setlist[itemize]{noitemsep, topsep=0pt}
%\setlist[enumerate]{noitemsep, topsep=0pt}


%%% Каждый раздел (\section) с новой страницы
\let\stdsection\section
\renewcommand\section{\newpage\stdsection}